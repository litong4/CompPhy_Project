In this section we discuss the properties of $L \times L$ Ising model with periodic boundary condition, 
which indicates that a spin on one boundary will interact with another spin on the opposite boundary. 
In other words, the summation over $<kl>$ in Eq. \ref{eq:hamiltonian} includes the following cases: 
\begin{equation}
k=(1,x),\ l=(L,x);\ k=(x,1),\ l=(x,L)
\end{equation}
where $(x,y)$ gives the coordinate of a spin in the $L \times L$ lattice. 

\subsection{The solution of $2 \times 2$ Ising model}\label{sec:2times2}
\begin{table}[tb]
	\centering
	\caption{All possible microscopic states of $2 \times 2$ Ising model. The energy is in the unit of $J$. }
	\begin{tabular}{cccc}
		\hline
		\hline 
		Number of spins up & Degeneracy & Energy & Magnetization \\ 
		\hline
		4 & 1 & -8 & 4 \\  
		3 & 4 & 0 & 2 \\ 
		2 & 4 & 0 & 0 \\ 
		2 & 2 & 8 & 0 \\ 
		1 & 4 & 0 & -2 \\ 
		0 & 1 & -8 & -4 \\ 
		\hline 
		\hline
	\end{tabular}
	\label{tab:2times2} 
\end{table}
For simplicity, we begin our discussion of the analytical solution from the $2 \times 2$ case. 
Table \ref{tab:2times2} shows all the possible microscopic states of  $2 \times 2$ Ising model. 
The corresponding partition function can be obtained by summing over all these microscopic states $\alpha$ 
\begin{equation}
Z=\sum_{\alpha}e^{-\beta E_\alpha}=2e^{8\beta}+2e^{-8\beta}+12=4\cosh\left(8\beta\right)+12\,,
\end{equation}
where $\beta=1/T$ and $T$ is temperature in the unit of $J$ (Boltzmann constant $k_B$ is absorbed into $T$). 
\par
Thus, the mean energy of the system (in the unit of $J$) is 
\begin{equation}
\langle E\rangle=-\frac{\partial \ln Z}{\partial \beta}=-\frac{8\sinh\left(8\beta\right)}{\cosh\left(8\beta\right)+3}\,. 
\end{equation}
The mean magnetization of the system is 
\begin{equation}
\langle|M|\rangle=\frac{1}{Z}\sum_{\alpha} |M_\alpha| e^{-\beta E_\alpha}=\frac{1}{Z}\left(8e^{8\beta}+4\right)
=\frac{2e^{8\beta}+1}{\cosh\left(8]\beta\right)+3}\,. 
\end{equation}
The heat capacity is 
\begin{equation}\label{eq:cv}
C_V=\frac{\partial \langle E\rangle}{\partial T}=-\beta^2\frac{\partial \langle E\rangle}{\partial \beta}
=\beta^2\frac{64\left[1+\cosh\left(8\beta\right)\right]}{\left[6+\cosh\left(8\beta\right)\right]^2}
=\beta^2\left(\langle E^2\rangle-\langle E\rangle^2\right)\,,
\end{equation}
where 
\begin{equation}
\langle E^2\rangle=\frac{1}{Z}\sum_{\alpha}E^2_\alpha e^{-\beta E_\alpha}=\frac{64\cosh\left(8\beta\right)}{\cosh\left(8\beta\right)+3}\,. 
\end{equation}
Similarly, we have 
\begin{equation}
\langle M^2\rangle=\frac{1}{Z}\sum_{\alpha} M^2_\alpha e^{-\beta E_\alpha}=\frac{8\left(e^{8\beta}+1\right)}{\cosh\left(8\beta\right)+3}\,, 
\end{equation}
and define susceptibility as 
\begin{equation}\label{eq:chi}
\chi=\frac{1}{\beta}\left(\langle M^2\rangle -\langle |M|\rangle ^2\right)\,. 
\end{equation}
Above results can be used to benchmark the Monte Carlo algorithm developed for 2D Ising model. 

\subsection{Properties of phase transition in 2D Ising model}\label{sec:transition}
One important aspect of the analytical solution of 2D Ising model is the phase transition at a nonzero critical temperature $T_C$. 
There is a sharp transition from nonzero $\langle|M|\rangle $ to $\langle|M|\rangle=0$ 
when temperature $T$ increases from $T<T_C$ to $T>T_C$. 
In this section we do not discuss the details of the analytical solution of 2D Ising model. 
Instead, we will show several important properties of phase transition in 2D Ising model, 
which are used to benchmark our simulation. 
\par
Near $T_C$ we can characterize the behavior of many physical quantities by a power law.
For 2D Ising model, the mean magnetization is given by
\begin{equation}
\langle M(T) \rangle \sim \left(T-T_C\right)^{\beta}\,,
\end{equation}
where $\beta=1/8$ is a critical exponent. A similar relation applies to heat capacity 
\begin{equation}
C_V(T) \sim \left|T_C-T\right|^{\alpha}\,,
\end{equation}
and the susceptibility
\begin{equation}
\chi(T) \sim \left|T_C-T\right|^{\gamma}\,,
\end{equation}
with $\alpha = 0$ and $\gamma = 7/4$.
Another important quantity is called correlation length, which is expected
to be of the order of the lattice spacing for $T>> T_C$. Because the spins
become more and more correlated as $T$ approaches $T_C$, the correlation
length increases as we get closer to the critical temperature. The divergent
behavior of $\xi$ near $T_C$ is 
\begin{equation}
\xi(T) \sim \left|T_C-T\right|^{-\nu}\,,
\label{eq:xi}
\end{equation}
with $\nu=1$. 
A second-order phase transition is characterized by a
correlation length which spans the whole system.
Since we are always limited to a finite lattice, $\xi$ will
be proportional with the size of the lattice. 
Through so-called finite size scaling relations
it is possible to relate the behavior at finite lattices with the 
results for an infinitely large lattice.
The critical temperature scales then as
\begin{equation}
T_C(L)-T_C(L=\infty) = aL^{-1/\nu}\,,
\label{eq:tc}
\end{equation}
where $a$ is a constant and $\nu$ is defined in Eq. \ref{eq:xi}. 
We set $T=T_C$ and obtain a mean magnetization
\begin{equation}
\langle M(T) \rangle \sim \left(T-T_C\right)^{\beta}
\rightarrow L^{-\beta/\nu}\,,
\label{eq:scale1}
\end{equation}
heat capacity
\begin{equation}
C_V(T) \sim \left|T_C-T\right|^{-\gamma} \rightarrow L^{\alpha/\nu},
\label{eq:scale2}
\end{equation}
and susceptibility
\begin{equation}
\chi(T) \sim \left|T_C-T\right|^{-\alpha} \rightarrow L^{\gamma/\nu}\,.
\label{eq:scale3}
\end{equation}
\par
The exact solution gives $\nu=1$ and $T_C=\frac{2}{\ln\left(1+\sqrt{2}\right)}\approx 2.269$ 
(in the unit of $J$) when $L\rightarrow\infty$. 
With the help of Eq. \ref{eq:tc}, we can use our simulation results of different $L$ to extract $T_C$ for $L\rightarrow\infty$, 
which will be done in Sec. \ref{results}. 
