In our work the Metropolis algorithm is adopted for the simulation of Ising model. 
This algorithm is based on the theory of Markov chain and detailed balance. 
In the example of Ising model, the transition rate between two microscopic states $i$ and $j$ should satisfy 
\begin{equation}\label{eq:Wfrac}
\frac{W(j \rightarrow i)}{W(i \rightarrow j)}=\frac{w_i}{w_j}=e^{-\beta\left(E_i-E_j\right)}\,,
\end{equation}
with $w_i=\frac{1}{Z}e^{-\beta E_i}$ 
to obtain an equilibrated (and detailed-balanced) probability distribution given by canonical ensemble. 
We can model $W(j \rightarrow i)$ as a product of the probability $T(j \rightarrow i)$ to make a transition from $j$ to $i$ 
and the probability $A(j \rightarrow i)$ to accept this transition, namely 
\begin{equation}
W(j \rightarrow i)=T(j \rightarrow i)A(j \rightarrow i)\,.
\end{equation}
In the simulation of Ising model, we have no insight of $T(j \rightarrow i)$, 
so we can simply assume that all the $T(j \rightarrow i)$ are the same. 
Then Eq. \ref{eq:Wfrac} can be rewritten as 
\begin{equation}\label{eq:Afrac}
\frac{A(j \rightarrow i)}{A(i \rightarrow j)}=e^{-\beta\left(E_i-E_j\right)}\,,
\end{equation}
which describes the probability of accepting a transition from $j$ to $i$ in our simulation. 
\par
\begin{algorithm}[tb]
	\caption{Metropolis algorithm for the simulation of Ising model. }
	\label{alg::metropolis}
	\KwIn{Size of the system $L$, temperature $T$, number of Monte Carlo cycles $MC$. }
	\KwOut{$\langle E \rangle$, $\langle E^2 \rangle$, $\langle |M| \rangle$, $\langle M^2 \rangle$, $C_V$ and $\chi$. } 
	Initialize spin lattice $a$ (in an ordered or random way)\;
	Calculate $E$, $E^2$, $|M|$, $M^2$ of the initial lattice\; 
	$E_{tot}=0,\,E^2_{tot}=0,\,|M|_{tot}=0,\,M^2_{tot}=0$\;
	\For{$i=1;i<=MC;i++$}
	{
		\For{$j=1;j<=L;j++$}
		{
			\For{$k=1;k<=L;k++$}
			{
				$r=$ a uniformly distributed random number in $[0,1]$\;
				Flip spin at position $(j,k)$\;
				Calculate the change of energy $\Delta E$\; 
				\If{$\Delta E < 0\ \mathrm{or}\ r<\exp\left(-\Delta E/T\right)$}
				{
					Accept this spin flip\; 
					Update $E$, $E^2$, $|M|$, $M^2$\;
				}
				\Else
				{
					Reverse this spin flip\;
				}
			Add $E$, $E^2$, $|M|$, $M^2$ to their corresponding ``tot" variables\; 
			}
		}
	}
	Calculate the average $\langle E \rangle$, $\langle E^2 \rangle$, $\langle |M| \rangle$, $\langle M^2 \rangle$ by dividing $L^2MC$\; 
	Calculate $C_V$ and $\chi$ by Eq. \ref{eq:cv} and \ref{eq:chi}\;
\end{algorithm}
Based on Eq. \ref{eq:Afrac} we can develop a simulation algorithm shown in Algorithm \ref{alg::metropolis}. 
The main idea is that we flip one spin every time and use a random number $r$, which distributes uniformly in $[0,1]$, to determine whether to accept this flip based on Eq. \ref{eq:Afrac}. 
If this flip reduces the energy or $r<\exp\left(-\Delta E/T\right)$ ($\Delta E$ is the energy change in the transition) we will accept. 
Physical quantities $\langle E \rangle$, $\langle E^2 \rangle$, $\langle |M| \rangle$, $\langle M^2 \rangle$ 
are obtained by averaging them in the Monte Carlo process. 
