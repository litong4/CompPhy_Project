Ising model is a simple but important model for the explanation of ferromagnetism. 
It assumes that atomic spins are located on a $N$-dimensional lattice grid, 
and these spins only have two discrete states, up ($\sigma=1$) or down ($\sigma=-1$). 
Only neighboring spins can interact with each other, and thus its Hamiltonian can be written as 
\begin{equation}\label{eq:hamiltonian}
H=-J\sum_{<kl>}^{N}\sigma_k\sigma_l\,,
\end{equation}
where $J>0$ and $<kl>$ indicates that we only sum over nearest neighbors. 
\par
In statistical physics, the equilibrium of a system described by Eq. \ref{eq:hamiltonian} should be solved using canonical ensemble, 
which involves a summation over all possible microscopic states to obtain the partition function. 
For $N$-dimensional Ising model with size $L$ in each dimension, 
the total number of all microscopic states is $2^{L^N}$, which cannot be simply summed over. 
The one- and two-dimensional cases has been solved analytically and they show several interesting properties of phase transition. 
But the exact solution of higher-dimensional Ising model is still unavailable, 
and thus numerical methods which avoid the summation over all possible states are required for the simulation of Ising model. 
\par
In this report, we discuss the Monte Carlo method for the simulation of two-dimensional (2D) Ising model 
and compare our results with the analytical solution. 
In Sec. \ref{theory} several properties shown in the analytical solution are discussed, for the comparison in Sec. \ref{results}. 
Sec. \ref{method} gives the outline of Monte Carlo algorithm we utilize for 2D Ising model, 
and then Sec. \ref{results} discusses the results of our simulation. 
Conclusions are given in Sec. \ref{conclude}. 
