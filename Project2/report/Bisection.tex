As we are usually interested in the symmetric tridiagonal eigenproblem, the bisection method can be applied. 
The eigen-polynomial of a n-by-n tridiagonal matrix
\begin{equation}
A=\left[\begin{array}{cccccc}   
    a_1 &    b_1    & 0   & \cdots  & \cdots  & 0\\  
    b_1 &    a_2    & b_2  & 0       & \cdots  & 0\\
    0 &    b_2    & a_3  & b_3       & \cdots  & 0\\ 
    \vdots &    \vdots    & \ddots  & \ddots  & \ddots  & 0\\ 
    0 &    \cdots    & \cdots  & b_{n-2}       & a_{n-1}  & b_{n-1}\\ 
    0 &    \cdots    & \cdots  & 0       & b_{n-1}  & a_{n}\\ 
\end{array}\right]
\end{equation} 
satisfies a general recurrence relation
\begin{equation}
	p_r(x)=(a_r-x)p_{r-1}(x)-b_{r-1}^2p_{r-2}(x)
\end{equation}
where $r=2:n$, $p_0(x)=1$ and $p_1(x)=a_1$.

The Sturm sequence property \cite{golub2012matrix} reveals the relation between eigenvalues neighboring eigen-polynomials. 
	The relation is 
\begin{equation}
	\lambda_r(A_r)<\lambda_{r-1}(A
	_{r-1})<\lambda_{r-1}(A
	_{r})<\cdots<\lambda_2(A
	_{r})<\lambda_{1}(A_
	{r-1})<\lambda_{1}(A_{r}).
\end{equation}.
Moreover, if $a(\alpha)$ denotes the number of sign changes in sequence
\[
	\left\{p_0(\alpha),p_1(\alpha),\cdots,p_n(\alpha)\right\},
\]
then $a(\alpha)$ equals the number of p's eigenvalues that are less than $\alpha$. 
Also the Gershgorin theorem fixes the lower and upper bounds of eigenvalues. 
The $\lambda_k(p)\in[y,z]$ where
\begin{equation}\label{bounds}
	y = \min_{1\le i \le n}a_i - |b_i| -|b_{i-1}|,\quad z = \max_{1\le i \le n}a_i + |b_i| + |b_{i-1}|.
\end{equation}
During the execution of the bisection method, information about the location of other eigenvalues is obtained.
By keeping track of this information it is possible to extract subintervals for every eigenvalues. 
See Ref. \cite{wilkinson1962calculation}.
Then, we can apply the bisection method for each subinterval to yield corresponding eigenvalue. 
In addition, in this method, we allocate space only for tridiagonal elements as others are not referred. 
It saves us a lot of memory which is crucial for huge size  calculations.
The implementation of the bisection method is illustrated in Algorithm \ref{BisectionAlgo}.
\begin{algorithm}[tb]
	\label{BisectionAlgo}
	\caption{The bisection method for diagonalization of symmetric tridiagonal matrix $A\in R^{n\times n}$. }
	\KwIn{Diagonal vector $\mathbf{a}$, off-diagonal vector $\mathbf{b}$, tolerence $tol$}
	\KwOut{Eigenvalues in vector $\mathbf{u}$ and corresponding eigenvectors in columns of matrix $S\in\mathbb{R}^{n\times n}$. }
	Calculate lower bound y and upper bound z
	via Eq. \ref{bounds};\\
	// Determine subintervals;\\
	Calculate $p_n(y), p_n(z), a(y)$ and $a(z)$;\\
	 \While{$a(z)-a(y) > 1$}
    {
		bisectional searching for bounds of subintervals and store them in vector $subint(n+1)$;
    }
    // Apply the bisection method to every subintervals;\\ 
    \For{$i=0;i<n;i++$}
	{
	$y = subint(i)$, $z = subint(i+1)$;\\
    \While{$rmax-rmin>tol$}
    {
		$x=(y+z)/2$;\\
		\If {$p_n(x)*p_n(y)<0$}{$z=x$;}
		\Else{$y=x$;}
    }
    $u(i) = x$;
    }
    // Calculate correlated eigenvectors;\\
    \For{$i=0;i<n;i++$}
    {
    solve $(A-u(i)I)S(i,:) = 0;$
    }
\end{algorithm}
